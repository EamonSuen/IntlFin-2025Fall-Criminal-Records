\documentclass{chinese-erj}

\title{隐匿的信号与显性的歧视:犯罪记录封存的非预期一般均衡}
\author{ 孙翊铭 \quad 2025212062}
\date{}
\header{隐匿的信号与显性的歧视:犯罪记录封存的非预期一般均衡}



\begin{document}

\maketitle

\begin{abstract}
本文基于劳动经济学与信息经济学的分析框架,探讨了中国2026年违法记录封存制度对劳动力市场均衡的潜在影响。通过整合 \textcite{alpert2018supply} 关于供给侧干预的替代效应与 \textcite{agan2018ban} 关于统计性歧视的理论机制,本文揭示了在信息摩擦环境下,单纯的行政屏蔽可能诱发的非预期后果。理论推演表明,中国特有的官民二元信息结构可能导致私营部门产生策略性怀疑,使无记录证明的信号信噪比显著下降。在缺乏可信甄别机制的情形下,雇主理性的风险规避行为将导致劳动力市场陷入混同均衡,并引致跨群体的要素替代效应。本文据此提出建立司法康复证书等高置信度信号机制的政策建议,以实现从混同均衡向分离均衡的帕累托改进。
\keywords{犯罪记录封存 \quad 信息摩擦 \quad 统计性歧视 \quad 混同均衡 \quad 要素替代}
\end{abstract}

\section{引言}
中共中央关于进一步全面深化改革、推进中国式现代化的决定》提出“建立轻微犯罪记录封存制度”,随后《治安管理处罚法》修订工作正式将这一议题纳入立法议程。在这一进程中,立法者试图通过制度供给降低社会改造边际成本的技术理性,与公众基于朴素正义观的“零容忍”直觉,呈现出不可调和的背离。一方面,制度设计的初衷在于通过消除犯罪记录对就业的负面影响,修复前科人员的人力资本,从而在宏观上降低累犯率;另一方面,舆论场中关于“特权阶层获益”的叙事(如网络热议的“哪位少爷吸了”),反映出社会对制度被权力寻租异化的深刻担忧。

在劳动经济学视角下,这一冲突本质上是人力资本修复收益与信息不对称成本之间的权衡。理论上,封存制度通过切断负面信号的传递,能有效提升前科群体的劳动参与率。然而,强制性的信息屏蔽人为制造了严重的市场摩擦。在信息缺失的环境下,理性的雇主为了规避潜在风险,不得不基于特定群体(如特定地域、年龄或学历背景)的先验信念(Prior Beliefs)进行贝叶斯更新。这种统计性歧视(Statistical Discrimination)机制可能导致原本旨在保护少数人的政策,引发了针对更广泛群体的“株连效应”,即无辜者被迫承担了高风险个体的平均风险溢价。

本文旨在将这一中国当下的政策议题置于标准的经济学分析框架内。基于 \textcite{alpert2018supply} 关于供给侧干预的替代效应、\textcite{mueller2021diversion} 关于标签效应的实证发现以及 \textcite{agan2018ban} 的统计性歧视模型,本文分析了中国制度语境下的一般均衡后果。 本文的核心论点是,在缺乏配套认证体系的情形下,保护性政策可能反向恶化目标群体的就业环境。原因在于,中国独特的“官民二元”信息结构(即记录对官方透明、对民间屏蔽)可能引发市场严重的信任折价(Trust Discount)。若缺乏可信的信号甄别机制,单纯的“记录封存”不仅会导致低学历年轻男性群体陷入被均值定价的混同均衡(Pooling Equilibrium),更会诱发雇主进行要素替代(Factor Substitution)——即转而雇佣女性、年长劳动者或通过资本深化来替代该类高风险的劳动力。

\section{制度背景与特征事实}

随着修订后的《治安管理处罚法》于2026年正式实施,中国建立了一套具有鲜明“双轨制”特征的记录管理体系。针对未成年人及特定轻微违法,包括符合条件的首次吸毒行为,其违法记录对一般社会主体进行封存。这意味着在标准的入职背景调查中,该记录将不再显示,旨在消除市场准入的显性壁垒。同时,封存并不等同于消灭。公安机关、国家安全机关及监察机关在进行特定审查时,仍拥有完全的信息调取权限。此外,对于法律规定的特定行业记录依然保持可见。 这种设计试图在微观的人力资本保护与宏观的社会安全管控之间寻求平衡,但在实际运行中,它改变了劳动力市场的信息集结构。



与制度设计同样重要的是市场主体对该制度的反应函数。在中国当前的舆论环境中,公众对毒品问题持有极高的风险厌恶偏好。更为关键的是,舆论场中关于“特权阶层获益”的叙事反映了对执行程序公正性的担忧。 经济学直觉表明,当市场主体怀疑封存制度可能被寻租行为渗透时,他们对“无记录”信号的可信度评价会显著降低。这种严峻的信任赤字(Trust Deficit)构成了本文分析的关键约束条件:即我们讨论的是基于而是存在严重信息摩擦且缺乏完备信用背书的市场环境。这为后续章节分析统计性歧视的发生机制提供了现实土壤。

\section{理论机制与文献评述}
本节将已有文献中的实证发现梳理为三个核心机制,以分析供给侧干预和信息限制如何改变个体行为与市场均衡。
\subsection{供给侧限制与替代品的溢出效应}

政策制定者常试图通过切断特定“坏品”的供给来抑制其负外部性,但忽略替代品市场的交叉弹性往往导致意料之外的均衡结果。\textcite{alpert2018supply}为这一机制提供了经典的实证证据。 该研究利用2010年奥施康定(OxyContin)配方改良这一外生冲击,分析了毒品市场的供给侧干预效果。2010年的配方改良引入了防滥用机制,使其难以通过粉碎或溶解进行注射或吸入,从而显著增加了滥用成本。改良后的药物难以被滥用,这在本质上是对特定成瘾品实施了技术性的供给限制。实证结果表明,尽管奥施康定的滥用率下降,但需求并未消失,而是由于高替代弹性迅速溢出至更危险的替代品——海洛因。这种“挤出效应”导致海洛因过量死亡率显著上升。据\textcite{alpert2018supply}估算,2010年后美国海洛因死亡率激增幅度的80\%可归因于此次配方改良引发的替代效应。 

这一发现为理解犯罪信息管制提供了理论启示。若将犯罪记录视为劳动力市场上关于风险属性的信息供给,封存制度本质上实施了针对该信息的供给侧阻断。在需求侧,即雇主规避风险的动机未发生改变的前提下,单纯切断某一特定信号的供给,并不会消除甄别行为本身,而只会迫使雇主转向寻找地域、户籍等次优替代信号。这种信号替代机制正是导致统计性歧视泛滥的微观根源。

\subsection{标签效应与人力资本积累}

在完全竞争的劳动力市场中,犯罪记录构成了对个体生产力的负面信号,导致所谓的“伤疤效应”(Scarring Effect)。\textcite{mueller2021diversion}基于得克萨斯州哈里斯县的行政数据,通过断点回归设计(RDD)证实了消除这一负面信号的效率收益。 研究发现,通过“分流”(Diversion)制度避免初犯者背负重罪标签,能显著改善其长期经济表现。去标签化通过消除犯罪记录的“污名效应”(Stigma Effect),使得个体能够保留劳动力市场的准入资格,使其未来十年内的就业率提升了近50\%,并使再犯率减半。这一机制表明犯罪记录封存制度具有坚实的效率基础:通过阻断标签对人力资本的侵蚀,制度可以纠正因一次失误导致的终身生产力折损,实现帕累托改进。这构成了中国建立该制度的主要正当性来源。

\subsection{信息不对称与统计性歧视}


然而,当政策由“避免产生记录”转向“强制隐藏记录”时,信息经济学的逻辑发生了根本性的变化。\textcite{agan2018ban}  以及 \textcite{doleac2020unintended}对“Ban the Box”(BTB)政策的研究揭示了信息不对称环境下的统计性歧视(Statistical Discrimination)机制。在信息完全的情形下,雇主通过个体特征进行甄别(Screening);而在记录被强制封存的信息不完全情形下,雇主无法观测个体的真实风险类型 $\theta_i$。为了最大化期望效用,理性雇主不得不依赖可观测的群体特征 $X$(如种族、年龄、教育程度)来推断风险概率 $P(\theta|X)$。

两项研究均证实,这种贝叶斯推断导致了对高风险群体(如年轻黑人男性)的广泛歧视。即便该群体中的守法公民,也因无法自证清白而被迫承担了群体的“平均风险溢价”。这表明,缺乏配套信号机制的封存政策,虽然保护了前科人员,却可能将成本转嫁给了同一群体中的无辜者,加剧了劳动力市场中的统计性歧视,造成了严重的效率损失。


\section{理论推演与可检验假说}

基于上述文献构建的分析框架,本节将模型环境参数校准为中国2026年制度实施的具体情境,提出三个关于劳动力市场均衡变化的核心假说。这些假说为未来的实证研究提供了潜在方向。

\textcite{agan2018ban}的核心发现是,当显性信号被屏蔽,雇主会转向隐性特征进行贝叶斯推断。在美国,这一隐性特征主要映射为种族;而在中国同质化的族群结构中,统计性歧视将沿着地域与人力资本两个维度展开。若雇主无法查询具体的吸毒记录,理性的风险厌恶策略是提高对来自某些特定地区求职者的先验风险评估。这将导致歧视从个体上移至某些特定的籍贯,使得该地区所有清白的求职者面临更高的统计性排斥。同时,学历不仅代表生产力,更将被雇主视为“自律性”和“守法概率”的强代理变量。由于雇主无法在该群体内通过甄别剔除高风险个体,理性的最优策略是对该群体所有成员索取普遍的风险折价,迫使无辜者陷入与潜在犯罪者被均值定价的混同均衡(Pooling Equilibrium)。

\textbf{假说 1(统计性歧视的转移):}《治安管理处罚法》实施后,在控制个体特征不变(Ceteris Paribus)的情况下,来自毒品高发地区或低学历背景的年轻男性求职者,其简历回复率(Callback Rate)与起薪水平将显著下降,且该效应在对安全性要求较高的服务业岗位中更为显著。


中国制度设计的独特性在于构建了一种“官民二元”的信息结构。所谓的“犯罪记录封存”,本质上属于行政屏蔽模式,而非德国《联邦中央登记法》(Bundeszentralregistergesetz, BZRG)的“彻底删除”(Tilgung)或法国未成年人司法中实行的“前科消灭模式”(Criminal Record Expungement)。在后两个国家的制度安排中,记录在达到法定年限后将从国家数据库中物理移除,即便是执法机关在后续的一般性调查中亦无法获取。这种“彻底性遗忘”消除了信息不对称的根基——因为不存在被隐藏的真实信息。相比之下,中国的“双轨制”保留了完整的官方数据记录:同一条犯罪记录对政府(公安、国安及特殊岗位审查)是透明的,仅对一般社会主体(百姓、私营雇主)隐形。


在博弈论视角下,这种由于行政权限导致的信息不可得,与由于物理灭失导致的信息不存在,会诱发截然不同的雇主信念。前者极易让市场产生策略性怀疑,即雇主意识到官方可能掌握了该求职者的高风险证据,仅仅是由于行政壁垒而未予披露。

市场均衡取决于信号的信噪比。若公众普遍持有“特权阶层更容易通过寻租获得记录封存”的先验信念(Prior Belief),那么“无犯罪记录”这一信号在私营部门眼中的可信度将显著下降。根据 \textcite{akerlof1978market} 的柠檬市场模型,当买方(雇主)认为高质量商品(真实无前科者)与低质量商品(被封存的前科者)在统计上无法区分,且担忧后者通过寻租混入的比例不确定时,市场会索取普遍的信任折价。这意味着,若封存制度执行不透明,不仅难以有效保护前科人员,反而会污染记录清白的信号价值,导致劳动力市场对该群体整体雇佣意愿的边际下降。

\textbf{假说 2(双轨制下的信任折价):}《治安管理处罚法》实施后,由于私营部门面临更高的信息摩擦,相对于拥有完全信息调取权限的公共部门(如国企、公务员系统),私营部门对持有无记录证明求职者的雇佣意愿将呈现边际收缩,导致公私部门间针对同类劳动力的工资剪刀差扩大。


我们进一步将\textcite{alpert2018supply}关于毒品市场替代效应的逻辑映射至要素市场。在“官民二元”的信息结构下,私营部门面临的不仅是平均估值的下降,更是风险方差的剧增。由于官方掌握全信息而私营部门仅拥有被行政屏蔽的信号,雇主无法确定眼前的年轻男性求职者是真正的低风险类型,还是被制度强行混同的高风险类型。对于风险厌恶型的企业而言,这种由信息阻断带来的不确定性成本可能超过了劳动力的边际产出收益。为了规避这种由制度制造的信息却是,市场将自发寻求确定性更高的替代要素,从而引发替代效应。其一,雇主的劳动力需求将向低信息噪音群体转移,增加对犯罪率统计显著较低的女性或已有长期可追溯信用记录的年长劳动者的雇佣偏好,以替代年轻男性这一高信噪比群体。其二,在安保、物流、家政等对安全性敏感的服务行业,过高的信息甄别成本将改变要素相对价格,促使企业加速引入自动化设备或平台化算法管理,以资本的零道德风险替代人力的潜在合规风险。

\textbf{假说 3(要素替代效应):}《治安管理处罚法》实施后,高风险厌恶型企业将显著调整要素投入结构。表现为:(1)跨群体替代,即显著增加对女性或年长劳动者的雇佣比例;(2)资本替代劳动,即在安保、物流等行业加速引入自动化设备以替代年轻男性劳动力。

\section{结论与政策建议}
本文在劳动经济学分析框架下,思考了中国违法记录封存制度的潜在影响。理论推演表明,在缺乏完善信号机制的情形下,单纯的供给侧信息阻断可能诱发严重的统计性歧视,导致社会福利的非线性损失。为了避免保护弱者的初衷异化为伤害无辜的结局,必须在制度设计中引入可信的信号甄别机制,将劳动力市场从低效率的混同均衡引导至高效率的分离均衡。

第一,供给侧干预的溢出效应。 借鉴\textcite{alpert2018supply}的发现,单纯阻断“轻微犯罪记录”这一负面信号的供给,并不能消除雇主对风险的需求。相反,这会迫使雇主转向使用地域、学历等高噪音信号进行贝叶斯推断,导致统计性歧视泛滥。 第二,“官民二元”结构可能导致信任陷阱。中国独特的行政屏蔽模式保留了官方记录,这在博弈中引发了私营部门的策略性怀疑。若缺乏透明的执行机制,市场将对“无记录证明”索取普遍的信任折价,甚至通过资本替代劳动来规避不确定性风险。

本文认为,制度的核心不应是简单的封存,而是建立一套让改过自新者能够以可验证的成本发送可信信号的体系。第一,行司法康复证书制度(Certificate of Rehabilitation)。建议由司法行政部门牵头,建立标准化的康复认证体系。类似信用修复机制,申请者需通过长期且严格的考核。例如,连续3年的尿检阴性证明、社区服务时长、职业技能培训合格证等。这种高昂的努力成本是区分真正悔过者与潜在模仿者的关键,持有该证书的求职者有权向特定雇主展示,从而将自己从高风险群体中分离出来。这不仅保护了隐私,更向市场提供了比单纯无记录更具信息含量的正向信号。

第二,建立分级可见的白名单信用数据库。利用中国在数字治理方面的优势,构建基于大数据、区块链技术的防篡改信用信息系统。将封存记录的查询权限与社会信用代码挂钩。对于一般企业,系统仅反馈是否准入的信号,屏蔽具体案由;对于涉及公共安全的特定岗位,系统自动触发详细预警。同时,因公开封存与解封的算法规则,减少人为裁量权,消除公众对于特权寻租的先验担忧,重建市场对无记录信号的贝叶斯信任。



第三,引入第三方担保与风险分担机制。为了缓解雇主的不确定性厌恶,政府可引入市场化保险机制。为雇佣封存记录人员的企业提供政策性保险补贴。若发生员工再犯导致的连带赔偿责任,由保险基金先行赔付。这相当于通过公共财政对冲了雇主的预期风险方差,从而降低其对该群体的雇佣门槛。






\nocite{*}

\erjref
\printbibliography[heading=none]



\end{document}
